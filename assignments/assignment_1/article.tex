\documentclass[10pt]{amsart}

\usepackage[]{libertine} 
\usepackage[]{blindtext} 
\usepackage[]{enumerate} 
\usepackage[]{cleveref} 

\newcommand{\C}{\mathbb{C}}
\newcommand{\N}{\mathbb{N}}
\newcommand{\Rez}[1]{\mathrm{Re}(#1)}
\newcommand{\Imz}[1]{\mathrm{Im}(#1)}

\theoremstyle{definition}
\newtheorem{exercise}{Exercise}
\newcommand{\limh}{\lim_{h \to 0}}


\title[Mandatory assignment 1]{
  Mandatory assignment 1 \\ 
  \hrulefill\fbox{MAT2410}\hrulefill
}
\author{
  Ivar Haugal\o kken Stangeby
}

\date{\today}

\begin{document}

\maketitle

\begin{exercise}\hfill
  \begin{enumerate}[a)]
    \item We wish to show that a circle $C \subset \C$ can be expressed equivalently as the set
  of points $z \in \C$ that satisfy
  \begin{equation}
    \notag
    C = \left\{ z \in \C \mid z\bar{z} - \bar{z_0}z - z_0\bar{z} + |z_0|^2 - r^2 = 0 \, \right\},
  \end{equation}
  where the typical expression reads
  \begin{equation}
    \notag
    C = \left\{ z \in \C \mid |z - z_0| = r \right\}.
  \end{equation}
  Taking the expression $|z - z_0| = r$, squaring both sides and using the identity
  ${|w|^2 = w\bar{w}}$ for any complex number $w$ we achieve
  \begin{align}
    \label{e:orig}
    \begin{split}
    |z - z_0|^2 - r^2 &= (z - z_0)(\bar{z} - \bar{z_0}) - r^2 \\
                      &= z\bar{z} - \bar{z_0}z - z_0\bar{z} + |z_0|^2 - r^2 \\
                      &= 0.
    \end{split}
  \end{align}
  This is what we wanted to show.
    \item We now wish to show that the set of points $z \in \C$ satisfying
    \begin{equation}  
      \label{e:circ}
      \left| \frac{z - z_1}{z - z_2} \right| = k, 
    \end{equation}
    is a circle where $z_1, z_2 \in \C$ with $z_1 \neq z_2$ and $k > 0$. We also want to find the center and radius of the circle.

    Multiplying through by $|z - z_2|$ and squaring both sides we achieve
    \begin{align*}
      |z - z_1|^2 - k^2|z - z_2|^2 &= (z - z_1)(\bar{z} - \bar{z_1}) - k^2 \big( (z - z_2)(\bar{z} - \bar{z_2}) \big) \\
                                   &= z\bar{z} - z\bar{z_1} - z_1\bar{z} + z_1\bar{z_1} - k^2\big( z\bar{z} - z\bar{z_2} - z_2\bar{z} + z_2\bar{z_2}\big) \\
                                   &= (1 - k^2)z\bar{z} + (k^2\bar{z_2} - \bar{z_1})z + (k^2z_2 - z_1)\bar{z} + (|z_1|^2 - k^2|z_2|^2)
    \end{align*}
    Under the assumption that $k \neq 1$, we divide by $(1 - k^2)$. Now we define ${w = \left( k^2 z_2 - z_1 \right) / (1 - k^2)}$ and $\psi = k^2|z_2|^2 - |z_1|^2 $. We can then rewrite the above equations as
    \begin{align*}
      |z - z_1|^2 - k^2|z - z_2|^2 &= z\bar{z} - \bar{w}z - w\bar{z} + |w|^2 - (\psi + |w|^2) \\
                                   &= z\bar{z} - \bar{w}z - w\bar{z} + |w|^2 - r^2 = 0
    \end{align*}
    where $r^2 = \psi + |w|^2$. We have now rewritten \cref{e:circ} on the form \cref{e:orig}
    Hence our original equation describes a circle centered at $w$ with radius $r = \sqrt{\psi + |w|^2}$    

    \item If we assume $k = 1$, then the set described in \cref{e:circ} is just the line equidistant from $z_1$ and $z_2$. In other words, its the set of points that are just as far away from $z_1$ as $z_2$.
  \end{enumerate}
\end{exercise}
\clearpage
\begin{exercise}\hfill
  \begin{enumerate}[a)]
    \item We wish to solve the quadratic equation $z^2 + 4z + 16 = 0$ and write the solution in polar form.
      A simple application of the \emph{abc}-formula for quadratic equations yields
      \begin{equation}
        \notag
        z = \frac{-4 \pm \sqrt{4^2 - 4\cdot 16}}{2} = -2 \pm 2\sqrt{3}i.
      \end{equation}
  \end{enumerate} 
  Hence the two roots of the equations are given by
  \begin{align*}
    z_1 = -2 + 2\sqrt{3}i \quad \text{and} \quad z_2 = -2 - 2\sqrt{3}i.
  \end{align*}
  In order to find the polar form for these two numbers, we use the relations
  \begin{align*}
    r = \sqrt{\Rez{z}^2 + \Imz{z}^2}, && \cos\theta = \frac{\Rez{z}}{r}, && \sin\theta = \frac{\Imz{z}}{r}.
  \end{align*}
  These yield
  \begin{align*}
    z_1 = 4e^{2\pi i /3}, \\
    \intertext{and}
    z_2 = 4e^{4\pi i/ 3}.
  \end{align*}
  \item We can now solve the equation
    \begin{equation}
      \label{e:equation} 
      z^6 = \frac{i}{1-i} = -\frac{1}{2} + \frac{i}{2}.
    \end{equation}
    We are looking for the complex number $z$ such that when taken to the sixth
    power we achieve the complex number $-1/2 + i/2$. This has to be a number
    with one sixth of the argument and the 6'th root of the modulus of $-1/2 +
    i/2$.
    So, using the relations above, we find that
    \begin{equation}
      \notag
      |z^6| = \frac{\sqrt{2}}{2}, \quad \text{and} \quad \arg(z^6) = \frac{3\pi}{4}.
    \end{equation}
    Hence, our $z$ must be the number
    \begin{equation}
      \notag
      z = \sqrt[6]{\frac{\sqrt{2}}{2}}e^{\frac{\pi i}{8}}, 
    \end{equation}
    which when plugged back into \cref{e:equation} yields $-1/2 + i/2$.
\end{exercise}

\begin{exercise}\hfill
  \begin{enumerate}[a)]
    \item We wish to determine where in the complex plane the function
      \begin{equation}
        \notag
        f(z) = e^{z^2}
      \end{equation}
      We see that this is the composition of two functions $g$ and $h$, with
      $g(z) = e^z$ and $h(z) = z^2$. Hence $f = g \circ h$.  We see immediately
      that $g$ is holomorphic, since it is \emph{analytic} on $\C$ and Theorem
      2.6 tells us that it must then be holomorphic. We also see that $h$ is
      holomorphic, since any complex polynomial is holomorphic, by Proposition
      2.2. It then follows, again by Proposition 2.2, that $f$ is holomorphic, and infact \emph{entire}.
    \item We now wish to determine if the function
      \begin{equation}
        \notag
        f(z) = e^{\bar{z}}
      \end{equation}
      is holomorphic. Again, we see that this is the composition of two functions, but $\bar{z}$ is not holomorphic. Hence, we have to examine the limit.
      \begin{equation}
        \label{eq:lim}
        \lim_{h \to 0} \frac{e^{\bar{z} + \bar{h}} - e^{\bar{z}}}{h}.
      \end{equation}
      We let $h = h_1 + ih_2$, and rewrite the limit as
      \begin{align*}
        \lim_{h \to 0} \frac{e^{\bar{z} + \bar{h}} - e^{\bar{z}}}{h} &= e^{\bar{z}} \left( \limh \frac{e^{h_1+ih_2} - 1}{h} \right) \\
                                                                     &\overset{\frac{0}{0}}{=} e^{\bar{z}} \left( \frac{\limh }{denom} \right)
      \end{align*}
    
  \end{enumerate} 
\end{exercise}

\begin{exercise}\hfill
  \begin{enumerate}
    \item 
  \end{enumerate} 
\end{exercise}
\end{document}
