\documentclass[10pt]{amsart}

\usepackage[]{libertine} 
\usepackage[]{blindtext} 
\usepackage[]{enumerate} 
\usepackage[]{cleveref} 

\newcommand{\C}{\mathbb{C}}
\newcommand{\N}{\mathbb{N}}

\theoremstyle{definition}
\newtheorem{exercise}{Exercise}


\title{
  Mandatory assignment 1 \\ 
  \hrulefill\fbox{MAT2410}\hrulefill
}
\author{
  Ivar Haugal\o kken Stangeby
}

\date{\today}

\begin{document}

\maketitle

\begin{exercise}\hfill
  \begin{enumerate}[a)]
    \item We wish to show that a circle $C \subset \C$ can be expressed equivalently as the set
  of points $z \in \C$ that satisfy
  \begin{equation}
    \notag
    C = \left\{ z \in \C \mid z\bar{z} - \bar{z_0}z - z_0\bar{z} + |z_0|^2 - r^2 = 0 \, \right\},
  \end{equation}
  where the typical expression reads
  \begin{equation}
    \notag
    C = \left\{ z \in \C \mid |z - z_0| = r \right\}.
  \end{equation}
  Taking the expression $|z - z_0| = r$, squaring both sides and using the identity
  ${|w|^2 = w\bar{w}}$ for any complex number $w$ we achieve
  \begin{align}
    \label{e:orig}
    \begin{split}
    |z - z_0|^2 - r^2 &= (z - z_0)(\bar{z} - \bar{z_0}) - r^2 \\
                      &= z\bar{z} - \bar{z_0}z - z_0\bar{z} + |z_0|^2 - r^2 \\
                      &= 0.
    \end{split}
  \end{align}
  This is what we wanted to show.
    \item We now wish to show that the set of points $z \in \C$ satisfying
    \begin{equation}  
      \label{e:circ}
      \left| \frac{z - z_1}{z - z_2} \right| = k, 
    \end{equation}
    is a circle where $z_1, z_2 \in \C$ with $z_1 \neq z_2$ and $k > 0$. We also want to find the center and radius of the circle.

    Multiplying through by $|z - z_2|$ and squaring both sides we achieve
    \begin{align*}
      |z - z_1|^2 - k^2|z - z_2|^2 &= (z - z_1)(\bar{z} - \bar{z_1}) - k^2 \big( (z - z_2)(\bar{z} - \bar{z_2}) \big) \\
                                   &= z\bar{z} - z\bar{z_1} - z_1\bar{z} + z_1\bar{z_1} - k^2\big( z\bar{z} - z\bar{z_2} - z_2\bar{z} + z_2\bar{z_2}\big) \\
                                   &= (1 - k^2)z\bar{z} + (k^2\bar{z_2} - \bar{z_1})z + (k^2z_2 - z_1)\bar{z} + (|z_1|^2 - k^2|z_2|^2)
    \end{align*}
    Under the assumption that $k \neq 1$, we divide by $(1 - k^2)$. Now we define ${w = \left( k^2 z_2 - z_1 \right) / (1 - k^2)}$ and $\psi = k^2|z_2|^2 - |z_1|^2 $. We can then rewrite the above equations as
    \begin{align*}
      |z - z_1|^2 - k^2|z - z_2|^2 &= z\bar{z} - \bar{w}z - w\bar{z} + |w|^2 - (\psi + |w|^2) \\
                                   &= z\bar{z} - \bar{w}z - w\bar{z} + |w|^2 - r^2
    \end{align*}
    where $r^2 = \psi + |w|^2$. We have now rewritten \cref{e:circ} on the form \cref{e:orig}
    Hence our original equation describes a circle centered at $w$ with radius $r = \sqrt{\psi + |w|^2}$    

    \item If we assume $k = 1$, then the set described in \cref{e:circ} is just the line equidistant from $z_1$ and $z_2$. In other words, its the set of points that are just as far away from $z_1$ as $z_2$.
  \end{enumerate}
\end{exercise}
\end{document}
